\documentclass[12pt]{article}
\usepackage[spanish]{babel}
\usepackage{amsmath}
\usepackage{amssymb}
\usepackage{graphicx}
\usepackage{ragged2e}
\usepackage{enumitem}
\usepackage{fancyvrb}
\usepackage[official]{eurosym}
\usepackage[utf8x]{inputenc} 
\graphicspath{ {images/} }
\begin{document}
\begin{figure}[t]
\title{Taller 1}
\includegraphics[{width=3cm, height=4cm}]{logouv.jpg}
\end{figure}
\title{Universidad del Valle \\ 
		Escuela de Ingeniería de Sistemas y Computación \\
		Programación por Restricciones \\
		Desarrollo: Taller Modelamiento e Implementación CSPs}
\author{Juan Marcos Caicedo Mejía (1730504-3743)}
\date{2020}
\maketitle

\newpage

\begin{center}
\large
\textbf{PARTE 1 (CSPs)}
\end{center}

\begin{flushleft}
\textbf{1. Sudoku}
\end{flushleft}

\begin{itemize}
\item Variables:
\begin{itemize}
\item \begin{verbatim}
subcuadricula
\end{verbatim} Representa la dimensión de cada subcuadricula. En un Sudoku común y corriente esto tiene un valor de 3. 
\item \begin{verbatim}
cuadricula
\end{verbatim} Representa la dimensión de toda la cuadricula grande (la que contiene las subcuadriculas). Al igual que la variable anterior, en un Sudoku común esta variable tiene el valor de 9.  
\item \begin{verbatim}
entrada
\end{verbatim} Representa el tablero de Sudoku inicial que posee algunos números preestablecidos. En esta matriz, el 0 representa una celda vacía.  
\item \begin{verbatim}
salida
\end{verbatim} Representa el tablero variable sobre el cual las restricciones tomarán acción y construirán la solución del Sudoku.
\item \begin{verbatim}
subcuadricula_uno
\end{verbatim} Es un array que contiene los 9 números correspondientes a la primera subcuadricula
\item \begin{verbatim}
subcuadricula_dos
\end{verbatim} Es un array que contiene los 9 números correspondientes a la segunda subcuadricula
\item \begin{verbatim}
subcuadricula_tres
\end{verbatim} Es un array que contiene los 9 números correspondientes a la tercera subcuadricula
\item \begin{verbatim}
subcuadricula_cuatro
\end{verbatim} Es un array que contiene los 9 números correspondientes a la cuarta subcuadricula
\item \begin{verbatim}
subcuadricula_cinco
\end{verbatim} Es un array que contiene los 9 números correspondientes a la quinta subcuadricula
\item \begin{verbatim}
subcuadricula_seis
\end{verbatim} Es un array que contiene los 9 números correspondientes a la sexta subcuadricula
\item \begin{verbatim}
subcuadricula_siete
\end{verbatim} Es un array que contiene los 9 números correspondientes a la séptima subcuadricula
\item \begin{verbatim}
subcuadricula_ocho
\end{verbatim} Es un array que contiene los 9 números correspondientes a la octava subcuadricula
\item \begin{verbatim}
subcuadricula_nueve
\end{verbatim} Es un array que contiene los 9 números correspondientes a la novena subcuadricula
\end{itemize}
\item Dominios:
\begin{itemize}
\item \begin{verbatim}
dimension_cuadricula = 1..cuadricula
\end{verbatim}
Este dominio es útil para poder acceder a todas las celdas de la cuadrícula grande, pues es el que demarca que los números que estén en el tablero solo sean del 1 al 9. 
\end{itemize}
\item Restricciones:
\begin{itemize}
\item Una primera restricción que solo sirve para poder copiar los datos del tablero inicial del Sudoku consiste en que si alguna celda en el tablero de la variable de entrada contiene un número mayor que 0, este mismo número se copie en la misma ubicación en el tablero de salida, si no es así, es decir, hay una celda con 0, se omite su valor:
\begin{equation*}
\forall i,j \in \{1,9\} \mid entrada(i,j) > 0 \therefore salida(i,j) = entrada(i,j)
\end{equation*}
\item Luego, una restricción importante es para garantizar que los números de todas las filas sean distintos:
\begin{equation*}
\forall i,j \in \{1,9\} \mid
\end{equation*}
\begin{equation*}
x_{1,1} \neq x_{1,2} \neq x_{1,3} \neq ... \neq x_{1,9}
\end{equation*}
\begin{equation*}
\land
\end{equation*}
\begin{equation*}
x_{2,1} \neq x_{2,2} \neq x_{2,3} \neq ... \neq x_{2,9}
\end{equation*}
\begin{equation*}
\land
\end{equation*}
\begin{equation*}
...
\end{equation*}
\begin{equation*}
x_{9,1} \neq x_{9,2} \neq x_{9,3} \neq ... \neq x_{9,9}
\end{equation*}
\item Luego, una restricción importante es para garantizar que los números de todas las columnas sean distintos:
\begin{equation*}
\forall i,j \in \{1,9\} \mid
\end{equation*}
\begin{equation*}
x_{1,1} \neq x_{2,1} \neq x_{3,1} \neq ... \neq x_{9,1}
\end{equation*}
\begin{equation*}
\land
\end{equation*}
\begin{equation*}
x_{1,2} \neq x_{2,2} \neq x_{3,2} \neq ... \neq x_{2,9}
\end{equation*}
\begin{equation*}
\land
\end{equation*}
\begin{equation*}
...
\end{equation*}
\begin{equation*}
x_{1,9} \neq x_{2,9} \neq x_{3,9} \neq ... \neq x_{9,9}
\end{equation*}
Donde $x_{i,j}$ es el número de la celda en la posición $(i,j)$.
\item Luego, usando las variables de: \begin{verbatim}
subcuadricula_uno
\end{verbatim} hasta \begin{verbatim}
subcuadricula_nueve
\end{verbatim}
Se impusieron las restricciones que permiten que en cada subcuadrícula no se repitan los números del 1 al 9. Por cada variable de subcuadricula (que contiene los numeros de dicha subcuadricula) se impuso la restricción de \begin{verbatim}
alldifferent
\end{verbatim}
Para la
\begin{verbatim}
subcuadricula_uno
\end{verbatim}
La restricción sería así:
\begin{equation*}
x_{1,1} \neq x_{1,2} \neq x_{1,3}
\end{equation*}
\begin{equation*}
\land
\end{equation*}
\begin{equation*}
x_{2,1} \neq x_{2,2} \neq x_{2,3}
\end{equation*}
\begin{equation*}
\land
\end{equation*}
\begin{equation*}
x_{3,1} \neq x_{3,2} \neq x_{3,3}
\end{equation*}
Para la
\begin{verbatim}
subcuadricula_dos
\end{verbatim}
La restricción sería así:
\begin{equation*}
x_{1,4} \neq x_{1,5} \neq x_{1,6}
\end{equation*}
\begin{equation*}
\land
\end{equation*}
\begin{equation*}
x_{2,4} \neq x_{2,5} \neq x_{2,6}
\end{equation*}
\begin{equation*}
\land
\end{equation*}
\begin{equation*}
x_{3,4} \neq x_{3,5} \neq x_{3,6}
\end{equation*}
Para la
\begin{verbatim}
subcuadricula_tres
\end{verbatim}
La restricción sería así:
\begin{equation*}
x_{1,7} \neq x_{1,8} \neq x_{1,9}
\end{equation*}
\begin{equation*}
\land
\end{equation*}
\begin{equation*}
x_{2,7} \neq x_{2,8} \neq x_{2,9}
\end{equation*}
\begin{equation*}
\land
\end{equation*}
\begin{equation*}
x_{3,7} \neq x_{3,8} \neq x_{3,9}
\end{equation*}
Para la
\begin{verbatim}
subcuadricula_cuatro
\end{verbatim}
La restricción sería así:
\begin{equation*}
x_{4,1} \neq x_{4,2} \neq x_{4,3}
\end{equation*}
\begin{equation*}
\land
\end{equation*}
\begin{equation*}
x_{5,1} \neq x_{5,2} \neq x_{5,3}
\end{equation*}
\begin{equation*}
\land
\end{equation*}
\begin{equation*}
x_{6,1} \neq x_{6,2} \neq x_{6,3}
\end{equation*}
Para la
\begin{verbatim}
subcuadricula_cinco
\end{verbatim}
La restricción sería así:
\begin{equation*}
x_{4,4} \neq x_{4,5} \neq x_{4,6}
\end{equation*}
\begin{equation*}
\land
\end{equation*}
\begin{equation*}
x_{5,4} \neq x_{5,5} \neq x_{5,6}
\end{equation*}
\begin{equation*}
\land
\end{equation*}
\begin{equation*}
x_{6,4} \neq x_{6,5} \neq x_{6,6}
\end{equation*}
Para la
\begin{verbatim}
subcuadricula_seis
\end{verbatim}
La restricción sería así:
\begin{equation*}
x_{4,7} \neq x_{4,8} \neq x_{4,9}
\end{equation*}
\begin{equation*}
\land
\end{equation*}
\begin{equation*}
x_{5,7} \neq x_{5,8} \neq x_{5,9}
\end{equation*}
\begin{equation*}
\land
\end{equation*}
\begin{equation*}
x_{6,7} \neq x_{6,8} \neq x_{6,9}
\end{equation*}
Para la
\begin{verbatim}
subcuadricula_siete
\end{verbatim}
La restricción sería así:
\begin{equation*}
x_{7,1} \neq x_{7,2} \neq x_{7,3}
\end{equation*}
\begin{equation*}
\land
\end{equation*}
\begin{equation*}
x_{8,1} \neq x_{8,2} \neq x_{8,3}
\end{equation*}
\begin{equation*}
\land
\end{equation*}
\begin{equation*}
x_{9,1} \neq x_{9,2} \neq x_{9,3}
\end{equation*}
Para la
\begin{verbatim}
subcuadricula_ocho
\end{verbatim}
La restricción sería así:
\begin{equation*}
x_{7,4} \neq x_{7,5} \neq x_{7,6}
\end{equation*}
\begin{equation*}
\land
\end{equation*}
\begin{equation*}
x_{8,4} \neq x_{8,5} \neq x_{8,6}
\end{equation*}
\begin{equation*}
\land
\end{equation*}
\begin{equation*}
x_{9,4} \neq x_{9,5} \neq x_{9,6}
\end{equation*}
Para la
\begin{verbatim}
subcuadricula_nueve
\end{verbatim}
La restricción sería así:
\begin{equation*}
x_{7,7} \neq x_{7,8} \neq x_{7,9}
\end{equation*}
\begin{equation*}
\land
\end{equation*}
\begin{equation*}
x_{8,7} \neq x_{8,8} \neq x_{8,9}
\end{equation*}
\begin{equation*}
\land
\end{equation*}
\begin{equation*}
x_{9,7} \neq x_{9,8} \neq x_{9,9}
\end{equation*}
\end{itemize}
\end{itemize}

\newpage

\begin{flushleft}
\textbf{2. Kakuro}
\end{flushleft}

\begin{itemize}
\item Variables:
\begin{itemize}
\item \begin{verbatim}
ancho
\end{verbatim}
Constante entera que se encarga de definir cuál será el ancho (en cuadrículas) del Kakuro.
\item \begin{verbatim}
alto
\end{verbatim}
Constante entera que se encarga de definir cuál será el alto (en cuadrículas) del Kakuro.
\item \begin{verbatim}
entrada_casillas
\end{verbatim}
Matriz constante que hace parte del input del Kakuro: esta matriz denota las casillas disponibles para rellenar (casillas blancas) como 0, y las casillas inhabilitadas para rellenar (generalmente las grises o que tienen diagonal) como -1.
\item \begin{verbatim}
entrada_casillas_left
\end{verbatim}
Matriz constante que hace parte del input del Kakuro: esta matriz denota las sumas horizontales en su respectiva casilla a la que pertenecen en el Kakuro como tal.
\item \begin{verbatim}
entrada_casillas_up
\end{verbatim}
Matriz constante que hace parte del input del Kakuro: esta matriz denota las sumas verticales en su respectiva casilla a la que pertenecen en el Kakuro como tal.
\item \begin{verbatim}
salida_casillas
\end{verbatim}
Matriz de variables de decisión que funciona como apoyo: pues sirve para decirle en qué casillas puede poner valores y en qué no, en la matriz en la cual si funcionan las restricciones.
\item \begin{verbatim}
salida_casillas_def
\end{verbatim}
Matriz de variables de decisión que hace el papel del resultado del Kakuro: sobre esta matriz se imponen las verdaderas restricciones de las reglas del Kakuro.
\end{itemize}
\item Dominios:
\begin{itemize}
\item \begin{verbatim}
dimension_ancho = 1..ancho
\end{verbatim}
Dominio que denota desde 1 hasta cuánto van las cuadrículas de todo el ancho del Kakuro.
\item \begin{verbatim}
dimension_alto = 1..alto
\end{verbatim}
Dominio que denota desde 1 hasta cuánto van las cuadrículas de todo el ancho del Kakuro.
\end{itemize}
\item Restricciones:
\begin{itemize}
\item La primera restricción consiste en rellenar la matriz de apoyo
\begin{verbatim}
salida_casillas
\end{verbatim}
Con los mismos elementos del array de
\begin{verbatim}
entrada_casillas
\end{verbatim}
Para poder saber qué casillas tenemos disponibles para rellenar y cuáles no (las inhabilitadas):
\begin{equation*}
\forall i \in dimension\_alto, j \in dimension\_ancho \land x \in \{-1, 0\}
\end{equation*}
\begin{verbatim}
entrada_casillas[i,j] = x
\end{verbatim}
\begin{equation*}
\Rightarrow 
\end{equation*}
\begin{verbatim}
salida_casillas[i,j] = x
\end{verbatim}
\item La segunda restricción consiste en que, usando la información del segundo array del input
\begin{verbatim}
entrada_casillas_left
\end{verbatim}
Si llega a encontrarse con un elemento que NO es -1, es decir, que significa que allí hay una suma horizontal, los elementos que se encuentren en la misma fila donde está dicho elemento, y que son 0, es decir, están disponibles para ser rellenados (esta información se consigue a partir de la matriz de apoyo salida\_casillas), entonces en la matriz definitiva de la solución salida\_casillas\_def, dichas celdas de 0 son las que deben se rellenadas con números del 1 al 9.
\begin{equation*}
\forall i \in dimension\_alto, j \in dimension\_ancho
\end{equation*}
\begin{verbatim}
entrada_casillas_left[i,j] != -1
\end{verbatim}
\begin{equation*}
\Rightarrow
\end{equation*}
\begin{equation*}
\forall k \in dimension\_ancho
\end{equation*}
\begin{verbatim}
salida_casillas[i,k] = 0
\end{verbatim}
\begin{equation*}
\Rightarrow
\end{equation*}
\begin{verbatim}
salida_casillas_def[i,k] 
\end{verbatim}
\begin{equation*}
\in \{1,9\}
\end{equation*}
\item La tercer restricción consiste en que, usando la información del tercer array del input
\begin{verbatim}
entrada_casillas_up
\end{verbatim}
Si llega a encontrarse con un elemento que NO es -1, es decir, que significa que allí hay una suma vertical, los elementos que se encuentren en la misma columna donde está dicho elemento, y que son 0, es decir, están disponibles para ser rellenados (esta información se consigue a partir de la matriz de apoyo salida\_casillas), entonces en la matriz definitiva de la solución salida\_casillas\_def, dichas celdas de 0 son las que deben se rellenadas con números del 1 al 9.
\begin{equation*}
\forall j \in dimension\_ancho, i \in dimension\_alto
\end{equation*}
\begin{verbatim}
entrada_casillas_up[i,j] != -1
\end{verbatim}
\begin{equation*}
\Rightarrow
\end{equation*}
\begin{equation*}
\forall k \in dimension\_alto
\end{equation*}
\begin{verbatim}
salida_casillas[k,i] = 0
\end{verbatim}
\begin{equation*}
\Rightarrow
\end{equation*}
\begin{verbatim}
salida_casillas_def[k,i] 
\end{verbatim}
\begin{equation*}
\in \{1,9\}
\end{equation*}
\item La cuarta restricción es la que hace la magia para que el Kakuro corresponda en sus sumas horizontales: Revisa mediante la función de agregación $forall$ la información del segundo array de input: 
\begin{verbatim}
entrada_casillas_left 
\end{verbatim}
Buscando, de manera similar a dos for anidados para recorrer una matriz en un lenguaje común, elemento a elemento, un elemento DISTINTO a -1, lo cúal significa que dicho elemento hace referencia a una suma horizontal que debe ser cumplida como restricción. \\ \\ 
De encontrar un elemento distinto a -1, va a verificar que la suma de los elementos que le siguen inmediatamente después en la fila a la ubicación del elemento que encontró (en el arreglo de output salida\_casillas\_def) sea igual al número que había encontrado que no era -1. 
\\ \\ 
Ahora, los Kakuro tienen distintas posibilidades en este momento, y son dos:
\begin{itemize}
\item Que dentro de las casillas siguientes al número que se debe rellenar haya una casilla inhabilitada, con lo cual tendrá casillas habilitadas (blancas) para cumplir la suma inmediatamente antes de dicha casilla inhabilitada. 
\item Que todas las casillas siguientes al número que debe dar la suma horizontal (dicho número se encuentra en una casilla inhabilitada para rellenar) estén habilitadas para rellenar, hasta llegar al ancho del Kakuro, es decir, que entre todas las casillas siguientes al número que se debe rellenar no haya ninguna inhabilitada para rellenar y esto llegue hasta el ancho del Kakuro.
\end{itemize}
\newpage
Lo que sucede en esta restricción es que, revisará que la suma de las casillas siguientes resulte en el número de la suma horizontal a cumplir, pero dependiendo de el caso que se de, revisará dicha suma de números en un rango distinto:
\begin{itemize}
\item \textbf{Caso 1: En las casillas siguientes al número que debe sumar, alguna casilla inhabilitada} \\ \\
Sea $\mathbb{S}$ el conjunto de todas las casillas inmediatamente siguientes a la casilla que contiene la suma horizontal:
\begin{equation*}
\mathbb{S} = \{s_0, s_1, s_2, ... , s_{ancho}\}
\end{equation*}
Por ejemplo, si el número que debe dar la suma horizontal se encuentra en una posición de columna $j$, es obligatorio que $s_0$ esté en la posición $j+1$.
\\ \\
Luego, sea $\mathbb{M}$ un subconjunto de $\mathbb{S}$ tal que
\begin{equation*}
m_0, m_1, m_2 ... m_n \in \mathbb{M} \Leftrightarrow m_0 = m_1 = m_2 = ... = m_n
\end{equation*}
\begin{equation*}
\land m_0 = -1 \land m_0, m_1, m_2 ... m_n \in \mathbb{S}
\end{equation*} 
Es decir, $\mathbb{M}$ es el subconjunto de los elementos de el conjunto $\mathbb{S}$ tales que sean iguales a -1. Es importante recalcar que de dicho conjunto, $m_0$ es el primer elemento que coincide en ser -1 en $\mathbb{S}$, por ende, es el primer elemento de $\mathbb{M}$.
\\ \\
De igual forma, el predicado $posicion(x)$ hace referencia a que, dado un elemento $x$ de una matriz, $posicion(x)$ devuelve la posicion de columna del elemento $x$.
\begin{equation*}
\forall i \in dimension\_alto, j \in dimension\_ancho
\end{equation*}
\begin{verbatim}
entrada_casillas_left[i,j] != -1
\end{verbatim}
\begin{equation*}
\Rightarrow
\end{equation*}
\begin{equation*}
\sum_{j+1}^{posicion(m_0)-1} x_{i} = entrada\_casillas\_left(i,j)
\end{equation*}
Dicha sumatoria denota que la sumatoria de todos los elementos siguientes en la fila al número que se encontró que no era -1, hasta una posición anterior al siguiente -1, es decir, la última casilla blanca habilitada para rellenar, todos deben sumar dicho número que fue encontrado que no era -1 (suma horizontal). Además deben ser distintos:
\begin{equation*}
\forall x \in \{j+1,posicion(m_0)-1\}
\end{equation*}
\begin{equation*}
x_{j+1} \neq x_{j+2} \neq x_{j+3} \neq ... \neq x_{posicion(m_0)-1}
\end{equation*}
\item \textbf{Caso 2: En las casillas siguientes al número que debe sumar, todas habilitadas hasta el ancho del Kakuro}
\begin{equation*}
\forall i \in dimension\_alto, j \in dimension\_ancho
\end{equation*}
\begin{verbatim}
entrada_casillas_left[i,j] != -1
\end{verbatim}
\begin{equation*}
\Rightarrow
\end{equation*}
\begin{equation*}
\sum_{j+1}^{ancho} x_{i} = entrada\_casillas\_left(i,j)
\end{equation*}
Dicha sumatoria denota que la sumatoria de todos los elementos siguientes en la fila al número que se encontró que no era -1, hasta una la última casilla habilitada que está ubicada en el ancho del Kakuro, deben sumar el número que se encontró que no era -1. Además deben ser distintos:
\begin{equation*}
\forall x \in \{j+1,ancho\}
\end{equation*}
\begin{equation*}
x_{j+1} \neq x_{j+2} \neq x_{j+3} \neq ... \neq x_{ancho}
\end{equation*}
\end{itemize}
\end{itemize}
\item La quinta restricción es la que hace la magia para que el Kakuro corresponda en sus sumas verticales: Revisa mediante la función de agregación $forall$ la información del tercer array de input: 
\begin{verbatim}
entrada_casillas_up 
\end{verbatim}
Buscando, de manera similar a dos for anidados para recorrer una matriz en un lenguaje común, elemento a elemento, un elemento DISTINTO a -1, lo cúal significa que dicho elemento hace referencia a una suma vertical que debe ser cumplida como restricción. \\ \\ 
De encontrar un elemento distinto a -1, va a verificar que la suma de los elementos que le siguen inmediatamente después en la columna a la ubicación del elemento que encontró (en el arreglo de output salida\_casillas\_def) sea igual al número que había encontrado que no era -1. 
\\ \\ 
Ahora, los Kakuro tienen distintas posibilidades en este momento, y son dos:
\begin{itemize}
\item Que dentro de las casillas siguientes al número que se debe rellenar haya una casilla inhabilitada, con lo cual tendrá casillas habilitadas (blancas) para cumplir la suma inmediatamente antes de dicha casilla inhabilitada. 
\item Que todas las casillas siguientes al número que debe dar la suma vertical (dicho número se encuentra en una casilla inhabilitada para rellenar) estén habilitadas para rellenar, hasta llegar al alto del Kakuro, es decir, que entre todas las casillas siguientes al número que se debe rellenar no haya ninguna inhabilitada para rellenar y esto llegue hasta el alto del Kakuro.
\end{itemize}
\newpage
Lo que sucede en esta restricción es que, revisará que la suma de las casillas siguientes resulte en el número de la suma vertical a cumplir, pero dependiendo de el caso que se de, revisará dicha suma de números en un rango distinto:
\begin{itemize}
\item \textbf{Caso 1: En las casillas siguientes al número que debe sumar, alguna casilla inhabilitada} \\ \\
Sea $\mathbb{S}$ el conjunto de todas las casillas inmediatamente siguientes a la casilla que contiene la suma vertical:
\begin{equation*}
\mathbb{S} = \{s_0, s_1, s_2, ... , s_{alto}\}
\end{equation*}
Por ejemplo, si el número que debe dar la suma horizontal se encuentra en una posición de fila $i$, es obligatorio que $s_0$ esté en la posición $i+1$.
\\ \\
Luego, sea $\mathbb{M}$ un subconjunto de $\mathbb{S}$ tal que
\begin{equation*}
m_0, m_1, m_2 ... m_n \in \mathbb{M} \Leftrightarrow m_0 = m_1 = m_2 = ... = m_n
\end{equation*}
\begin{equation*}
\land m_0 = -1 \land m_0, m_1, m_2 ... m_n \in \mathbb{S}
\end{equation*} 
Es decir, $\mathbb{M}$ es el subconjunto de los elementos de el conjunto $\mathbb{S}$ tales que sean iguales a -1. Es importante recalcar que de dicho conjunto, $m_0$ es el primer elemento que coincide en ser -1 en $\mathbb{S}$, por ende, es el primer elemento de $\mathbb{M}$.
\\ \\
De igual forma, el predicado $posicion(x)$ hace referencia a que, dado un elemento $x$ de una matriz, $posicion(x)$ devuelve la posicion de fila del elemento $x$.
\begin{equation*}
\forall j \in dimension\_ancho, i \in dimension\_alto
\end{equation*}
\begin{verbatim}
entrada_casillas_up[i,j] != -1
\end{verbatim}
\begin{equation*}
\Rightarrow
\end{equation*}
\begin{equation*}
\sum_{i+1}^{posicion(m_0)-1} x_{i} = entrada\_casillas\_up(i,j)
\end{equation*}
Dicha sumatoria denota que la sumatoria de todos los elementos siguientes en la columna al número que se encontró que no era -1, hasta una posición anterior al siguiente -1, es decir, la última casilla blanca habilitada para rellenar, todos deben sumar dicho número que fue encontrado que no era -1 (suma vertical). Además deben ser distintos:
\begin{equation*}
\forall x \in \{i+1,posicion(m_0)-1\}
\end{equation*}
\begin{equation*}
x_{i+1} \neq x_{i+2} \neq x_{i+3} \neq ... \neq x_{posicion(m_0)-1}
\end{equation*}
\item \textbf{Caso 2: En las casillas siguientes al número que debe sumar, todas habilitadas hasta el alto del Kakuro}
\begin{equation*}
\forall j \in dimension\_ancho, i \in dimension\_alto
\end{equation*}
\begin{verbatim}
entrada_casillas_up[i,j] != -1
\end{verbatim}
\begin{equation*}
\Rightarrow
\end{equation*}
\begin{equation*}
\sum_{i+1}^{alto} x_{i} = entrada\_casillas\_up(i,j)
\end{equation*}
Dicha sumatoria denota que la sumatoria de todos los elementos siguientes en la columna al número que se encontró que no era -1, hasta una la última casilla habilitada que está ubicada en el ancho del Kakuro, deben sumar el número que se encontró que no era -1. Además deben ser distintos:
\begin{equation*}
\forall x \in \{i+1,ancho\}
\end{equation*}
\begin{equation*}
x_{i+1} \neq x_{i+2} \neq x_{i+3} \neq ... \neq x_{ancho}
\end{equation*}
\end{itemize}
\end{itemize}

\newpage

\begin{flushleft}
\textbf{3. Secuencia Mágica}
\end{flushleft}

\begin{itemize}
\item Variables:
\begin{itemize}
\item \begin{verbatim}
n
\end{verbatim} La variable (ingresada por el usuario mediante el IDE o proveída por un archivo de datos .dzn), representa la longitud de la secuencia mágica.
\end{itemize}
\item Dominios:
\begin{itemize}
\item \begin{verbatim}
dominio = 0..n-1
\end{verbatim} Este dominio caracteriza dos cosas: el tamaño de la secuencia mágica (el tamaño del arreglo que la representa) y los posibles valores que toman los números dentro de la secuencia mágica.
\end{itemize}
\item Restricciones:
\begin{itemize}
\item Llamemos $ocurre$ a un predicado que recibe una lista $l$ y un número $x$, así, $ocurre(l,x)$ arroja el número de ocurrencias del número $x$ en la lista $l$. Otro predicado puede ser $posicion$ que dada una lista $l$ y un número $x$, retorna la posición del número $x$ en la lista $l$ (indexando desde 0). La restricción principal del problema consiste en que la secuencia mágica se caracteriza porque el número $i$ ocurre exactamente $x_{i}$ veces en la secuencia. Así, la restricción podría modelarse:
\begin{equation*}
\forall x \in l, ocurre(l, posicion(l, x)) = x
\end{equation*}
Esto quiere decir que el número de la posición de $x$ en la lista $l$ debe ocurrir $x$ veces en la lista $l$.
\item Una restricción redundante adicional, dicta que la suma de todos los números en la secuencia debe sumar el número $n$ (longitud de la secuencia):
\begin{equation*}
\sum_{i=0}^{n-1} x_{i} = n
\end{equation*}
Donde $x_{i}$ es el i-ésimo elemento de la secuencia mágica.
\item Otra restricción adicional, dice que la suma de cada elemento de la secuencia multiplicada por $i-1$ debe ser igual a 0:
\begin{equation*}
\sum_{i=0}^{n-1} x_{i}*(i-1) = 0
\end{equation*}
Donde $x_{i}$ es el i-ésimo elemento de la secuencia mágica.
\end{itemize}
\end{itemize}

\newpage

\begin{flushleft}
\textbf{4. Acertijo Lógico}
\end{flushleft}

\begin{itemize}
\item Variables:
\begin{itemize}
\item \begin{verbatim}
apellido_juan, apellido_oscar, apellido_dario
\end{verbatim}
Son las variables que corresponden a los apellidos de cada persona: Juan, Oscar y Dario. Los posibles valores que pueden tomar dichas variables son los elementos del dominio (tipo enumerado) \begin{verbatim}
APELLIDOS
\end{verbatim}
\item \begin{verbatim}
musica_juan, musica_oscar, musica_dario
\end{verbatim}
Son las variables que corresponden a la musica preferida de cada persona: Juan, Oscar y Dario. Los posibles valores que pueden tomar dichas variables son los elementos del dominio (tipo enumerado) \begin{verbatim}
MUSICA
\end{verbatim}
\item \begin{verbatim}
edad_juan, edad_oscar, edad_dario
\end{verbatim}
Son las variables que corresponden a la edad de cada persona: Juan, Oscar y Dario. Los posibles valores que pueden tomar dichas variables son los elementos del dominio (tipo enumerado) \begin{verbatim}
EDADES
\end{verbatim}
\end{itemize}
\item Dominios:
\begin{itemize}
\item \begin{verbatim}
APELLIDOS = {Gonzalez, Garcia, Lopez}
\end{verbatim}
Los posibles valores de los apellidos, dados por el enunciado del problema.
\item \begin{verbatim}
MUSICA = {Clasica, Pop, Jazz}
\end{verbatim}
Los posibles valores de la música preferida, dados por el enunciado del problema.
\item \begin{verbatim}
EDADES = 24..26
\end{verbatim}
Los posibles valores de las edades, de 24 a 26 años.
\end{itemize}
\item Restricciones:
Antes podríamos considerar los siguientes predicados:
\begin{enumerate}
\item $apellido(x,y) = $ La persona $x$ se apellida $y$
\item $musica(x,y) = $ La música favorita de la persona $x$ es $y$
\item $edad(x,y) = $ La persona $x$ tiene $y$ años
\end{enumerate}
$\rightarrow$ Restricciones:
\begin{itemize}
\item $\neg apellido(Juan, Gonzalez)$ \\ (Juan no se apellida González, esto se infiere al decir que Juan es mayor que González)
\item $\neg apellido(Oscar, Lopez)$ \\ (Oscar no se apellida López)
\item $\forall x \mid x \in APELLIDOS$ \\ $apellido(Juan,x_{i}) \land apellido(Oscar,x_{i+1}) \land apellido(Dario,x_{i+2}) \Rightarrow x_{i} \neq x_{i+1} \neq x_{i+2}$ \\ (Todos se deben apellidar distinto)
\item $\neg musica(Dario, Jazz)$ \\ (A Dario no le gusta el Jazz)
\item $\exists x \in APELLIDOS \land \exists y \in MUSICA \mid apellido(Oscar,x) \land x = Gonzalez \Rightarrow (musica(Oscar, y) \land y = Clasica)$
\item $\exists x \in APELLIDOS \land \exists y \in MUSICA \mid apellido(Dario,x) \land x = Gonzalez \Rightarrow (musica(Dario, y) \land y = Clasica)$ \\ (Como Juan no puede ser González, solo Oscar y Dario pueden apellidarse González, y si alguno de los dos se apellida González, su música favorita debe ser la clásica)
\item $\forall y \mid y \in MUSICA$ \\ $musica(Juan,y_{i}) \land musica(Oscar,y_{i+1}) \land musica(Dario,y_{i+2}) \Rightarrow y_{i} \neq y_{i+1} \neq y_{i+2}$ \\ (A cada uno le gusta un estilo musical distinto)
\item $edad(Oscar, 25)$ \\ (Oscar tiene 25 años)
\item $\exists x,y \in EDADES \mid$ \\ $ apellido(Oscar, Gonzalez) \Rightarrow edad(Juan,x) \land edad(Oscar,y) \land x > y$ \\ (Si Oscar se apellida Gonzalez, como Juan debe ser mayor que Gonzalez, Juan debería ser mayor que Oscar)
\item $\exists x,y \in EDADES \mid$ \\ $ apellido(Dario, Gonzalez) \Rightarrow edad(Juan,x) \land edad(Dario,y) \land x > y$ \\ (Si Dario se apellida Gonzalez, como Juan debe ser mayor que Gonzalez, Juan debería ser mayor que Dario)
\item $musica(Juan, Pop) \Rightarrow \exists x \in APELLIDOS \land \exists y \in EDADES \mid apellido(Juan, x) \land edad(Juan, y) \land x \neq Garcia \land y \neq 24$ \\
(Si Juan llegase a ser el fan de la música Pop, no puede apellidarse García ni tener 24 años)
\item $musica(Oscar, Pop) \Rightarrow \exists x \in APELLIDOS \land \exists y \in EDADES \mid apellido(Oscar, x) \land edad(Oscar, y) \land x \neq Garcia \land y \neq 24$ \\
(Si Oscar llegase a ser el fan de la música Pop, no puede apellidarse García ni tener 24 años)
\item $musica(Dario, Pop) \Rightarrow \exists x \in APELLIDOS \land \exists y \in EDADES \mid apellido(Dario, x) \land edad(Dario, y) \land x \neq Garcia \land y \neq 24$ \\
(Si Dario llegase a ser el fan de la música Pop, no puede apellidarse García ni tener 24 años)
\end{itemize}
\end{itemize}

\newpage

\begin{flushleft}
\textbf{5. Ubicación de personas en una reunión}
\end{flushleft}

\begin{itemize}
\item Variables:
\begin{itemize}
\item \begin{verbatim}
n
\end{verbatim}
Denota el número $n$ de personas en la reunión
\item \begin{verbatim}
cantidad_nexts
\end{verbatim}
Denota el número de restricciones de vecindad <<next>>, es decir, cuantas duplas hay en el array de next
\item \begin{verbatim}
cantidad_separates
\end{verbatim}
Denota el número de restricciones de no vecindad <<separate>>, es decir, cuantas duplas hay en el array de separate
\item \begin{verbatim}
cantidad_distancias
\end{verbatim}
Denota el número de restricciones de distancia entre personas <<distancia>>, es decir, cuantas tripletas hay en el array de distancia
\item \begin{verbatim}
entrada
\end{verbatim}
Es el arreglo que contiene las personas que saldrán en la foto. Todos sus elementos deben pertenecer al dominio del conjunto enumerado
\begin{verbatim}
PERSONAS
\end{verbatim}
\item \begin{verbatim}
next
\end{verbatim}
Es el arreglo que contiene las restricciones de proximidad entre una persona en la foto y otra. (cada dupla significa que una persona debe estar al lado de la otra)
\item \begin{verbatim}
separate
\end{verbatim}
Es el arreglo que contiene las restricciones de NO proximidad entre una persona en la foto y otra. (cada dupla significa que una persona no puede estar al lado de la otra)
\item \begin{verbatim}
distancia
\end{verbatim}
Es el arreglo que contiene las restricciones sobre distancia de separación entre una persona u otra. (cada tripleta significa que la primera persona debe estar separada de la otra persona por máximo un determinado número de personas)
\item \begin{verbatim}
salida
\end{verbatim}
Es el arreglo que contiene las variables de decisión sobre las cuales se aplican las restricciones apropiadas. Sus elementos también pertenecen al conjunto enumerado de
\begin{verbatim}
PERSONAS
\end{verbatim}
\end{itemize}
\item Dominios:
\begin{itemize}
\item \begin{verbatim}
PERSONAS
\end{verbatim}
El conjunto de las posibles personas que saldrán en la foto
\item \begin{verbatim}
DOMINIO1 = 1..n
\end{verbatim}
Dominio de 1 hasta el número $n$ de personas en la foto
\item \begin{verbatim}
DOMINIO2 = 1..cantidad_nexts
\end{verbatim}
Dominio que es útil para saber cuántas restricciones de next hay
\item \begin{verbatim}
DOMINIO3 = 1..cantidad_separates
\end{verbatim}
Dominio que es útil para saber cuántas restricciones de separate hay
\item \begin{verbatim}
DOMINIO4 = 1..cantidad_distancias
\end{verbatim}
Dominio que es útil para saber cuántas restricciones de distancia hay
\end{itemize}
\item Restricciones:
\begin{itemize}
\item Todas las personas en la fila de la foto deben ser distintas:
\begin{equation*}
\forall x \mid x \in PERSONAS \land x_{0}, x_{1}, ..., x_{i-2}, x_{i-1} \in salida
\end{equation*}
\begin{equation*}
x_{0} \neq x_{1} \neq ... \neq x_{i-2} \neq x_{i-1}
\end{equation*}
\item Restricciones next:
\begin{equation*}
\forall i \in DOMINIO1, j \in DOMINIO2
\end{equation*}
\begin{equation*}
salida(i) = entrada(next(j,1)) \land i = 1 \Rightarrow salida(i+1) = entrada(next(j,2))
\end{equation*}
\begin{equation*}
salida(i) = entrada(next(j,1)) \land i = 8 \Rightarrow salida(i-1) = entrada(next(j,2))
\end{equation*}
\begin{equation*}
salida(i) = entrada(next(j,1)) \land (i \neq 1 \land i \neq 8) \Rightarrow
\end{equation*}
\begin{equation*}
salida(i+1) = entrada(next(j,2)) \lor salida(i-1) = entrada(next(j,2))
\end{equation*}
\item Restricciones separate:
\begin{equation*}
\forall i \in DOMINIO1, j \in DOMINIO3
\end{equation*}
\begin{equation*}
salida(i) = entrada(separate(j,1)) \land i = 1 \Rightarrow salida(i+1) \neq entrada(separate(j,2))
\end{equation*}
\begin{equation*}
salida(i) = entrada(separate(j,1)) \land i = 8 \Rightarrow salida(i-1) \neq entrada(separate(j,2))
\end{equation*}
\begin{equation*}
salida(i) = entrada(separate(j,1)) \land (i \neq 1 \land i \neq 8) \Rightarrow
\end{equation*}
\begin{equation*}
salida(i+1) \neq entrada(separate(j,2)) \lor salida(i-1) \neq entrada(separate(j,2))
\end{equation*}
\item Restricciones distancia:
\begin{equation*}
\forall i,k \in DOMINIO1, j \in DOMINIO4
\end{equation*}
\begin{equation*}
salida(i) = entrada(distancia(j,1)) \land salida(k) = entrada(distancia(j,2)) \Rightarrow
\end{equation*}
\begin{equation*}
|i-k| \leq distancia(j,3)+1
\end{equation*}
\end{itemize}
\end{itemize}

\newpage

\begin{center}
\large
\textbf{PARTE 2 (COPs)}
\end{center}

\begin{flushleft}
\textbf{7. Construcción de Transformadores}
\end{flushleft}

\begin{enumerate}[label=\alph*]
\item Tabla que representa los datos del problema:


\begin{table}[h]
\begin{tabular}{|l|l|l|l|}
\hline
                  & Toneladas de material ferromagnético & Horas de trabajo & Precio Venta \\ \hline
Transformador \#1 & 2                                                           & 7                                      & 120.000 USD \\ \hline
Transformador \#2 & 1                                                           & 8                                      & 80.000 USD \\ \hline
\end{tabular}
\end{table}

\item Modelo COP para el problema utilizando los valores presentados en la tabla:

\begin{itemize}
\item Variables: 
\begin{itemize}
\item \begin{verbatim}
cantidad_toneladas_material_ferromagnetico = 6;
\end{verbatim}
Es una constante, corresponde a la cantidad de toneladas de material ferromagnético disponibles, que en el caso del problema, equivale a 6.
\item \begin{verbatim}
cantidad_horas_de_trabajo = 28;
\end{verbatim}
Es una constante, corresponde a la cantidad de horas de trabajo disponibles, que en el caso del problema, equivale a 28.
\item \begin{verbatim}
cantidad_toneladas_t1 = 2;
\end{verbatim}
Es una constante, corresponde a la cantidad de toneladas de material ferromagnético que se requieren para fabricar una unidad del Transformador \#1, en el problema equivale a 2.
\item \begin{verbatim}
cantidad_toneladas_t2 = 1;
\end{verbatim}
Es una constante, corresponde a la cantidad de toneladas de material ferromagnético que se requieren para fabricar una unidad del Transformador \#2, en el problema equivale a 1.
\item \begin{verbatim}
cantidad_horas_t1 = 7;
\end{verbatim}
Es una constante, corresponde a la cantidad de horas de trabajo que se requieren para fabricar una unidad del Transformador \#1, en el problema equivale a 7.
\item \begin{verbatim}
cantidad_horas_t2 = 8;
\end{verbatim}
Es una constante, corresponde a la cantidad de horas de trabajo que se requieren para fabricar una unidad del Transformador \#2, en el problema equivale a 8.
\item \begin{verbatim}
precio_venta_t1 = 120000;
\end{verbatim}
Es una constante, corresponde al precio de venta al público para el Transformador \#1, que en el problema equivale a 120.000
\item \begin{verbatim}
precio_venta_t2 = 80000;
\end{verbatim}
Es una constante, corresponde al precio de venta al público para el Transformador \#2, que en el problema equivale a 80.000
\item \begin{verbatim}
t1
\end{verbatim}
Variable de decisión que corresponde a cuántas unidades de Transformador \#1 se fabricarán.
\item \begin{verbatim}
t2
\end{verbatim}
Variable de decisión que corresponde a cuántas unidades de Transformador \#2 se fabricarán.
\item \begin{verbatim}
cantidad_toneladas_material_ferromagnetico_utilizadas = 
\end{verbatim} 
\begin{verbatim}
cantidad_toneladas_t1*t1 + cantidad_toneladas_t2*t2;
\end{verbatim}
Variable de decisión que corresponde al número total de toneladas de material ferromagnético utilizadas, según el número de transformadores de cada tipo fabricados.
\item \begin{verbatim}
cantidad_horas_de_trabajo_utilizadas = cantidad_horas_t1*t1 
\end{verbatim}
\begin{verbatim}
+ cantidad_horas_t2*t2;
\end{verbatim}
Variable de decisión que corresponde al número total de horas de trabajo utilizadas, según el número de transformadores de cada tipo fabricados.
\end{itemize}
\item Dominio: 
\begin{itemize}
\item \begin{equation*}
t_{1}, t_{2} \in \mathbb{Z}^{+} \cup \{0\}
\end{equation*}
Dominio para la cantidad de transformadores a fabricar $t_{1}$ y $t_{2}$ (cada uno). La cantidad de transformadores a fabricar debe ser entera y no negativa (incluyendo el cero).
\end{itemize}
\item Restricciones: 
\begin{itemize}
\item Restricciones de no negatividad:
\begin{equation*}
t_{1} \geq 0 \land t_{2} \geq 0
\end{equation*}
\item La cantidad de toneladas de material ferromagnético utilizadas no debe superar la cantidad con la que contamos al inicio:
\begin{equation*}
cantidad\_toneladas\_material\_ferromagnetico\_utilizadas \leq
\end{equation*}
\begin{equation*}
cantidad\_toneladas\_material\_ferromagnetico
\end{equation*}
\item La cantidad de horas de trabajo utilizadas no debe superar la cantidad con la que contamos al inicio:
\begin{equation*}
cantidad\_horas\_de\_trabajo\_utilizadas \leq
\end{equation*}
\begin{equation*}
cantidad\_horas\_de\_trabajo
\end{equation*}
\end{itemize}
\item Función objetivo:
\begin{itemize}
\item La función objetivo a \textbf{maximizar} es multiplicar el precio de venta el Transformador \#1 por la cantidad de transformadores $t_{1}$ fabricadas, más la multiplicación del precio de venta el Transformador \#2 por la cantidad de transformadores $t_{2}$:
\begin{equation*}
f = precio\_venta\_t_{1} * t_{1} + precio\_venta\_t_{2} * t_{2}
\end{equation*}
\end{itemize}
\end{itemize}

\item Presente la implementación del modelo anterior utilizando código MiniZinc: \textbf{se encuentra en el archivo transformadores.mzn}

\item Proponga un modelo COP genérico para $n$ transformadores asumiendo que los valores en la tabla
pueden cambiar:

\begin{justify}
$n$ transformadores:
\end{justify}

\begin{itemize}
\item \textbf{Entradas}:
\begin{itemize}
\item Una lista de tamaño $n$ que denota los precios de cada transformador:
\begin{equation*}
<p_{0}, p_{1}, p_{2}, ..., p_{n}>
\end{equation*}
Donde $p_{i}$ denota el precio a la venta del público del transformador $i$-ésimo
\item Una lista de tamaño $n$ que denota la cantidad de toneladas de material ferromagnético necesarias para fabricar cada transformador:
\begin{equation*}
<m_{0}, m_{1}, m_{2}, ..., m_{n}>
\end{equation*}
Donde $m_{i}$ denota la cantidad de toneladas de material ferromagnético necesario para fabricar el transformador $i$-ésimo
\item Una lista de tamaño $n$ que denota la cantidad de horas de trabajo necesarias para fabricar cada transformador:
\begin{equation*}
<h_{0}, h_{1}, h_{2}, ..., h_{n}>
\end{equation*}
Donde $h_{i}$ denota la cantidad de horas de trabajo necesarias para fabricar el transformador $i$-ésimo
\item Cantidad de toneladas de material ferromagnético inicial disponibles para utilizar $M$
\item Cantidad de horas de trabajo disponibles para utilizar $H$
\end{itemize}
\item \textbf{Salida}:
\begin{itemize}
\item Lista de tamaño $n$ cuyos elementos $x_{0}, x_{1}, x_{2}, ..., x_{n} \in \mathbb{Z}^{+} \cup \{0\}$ tal que:
\begin{equation*}
\sum_{i=0}^{n} x_{i} * p_{i}
\end{equation*}
sea \textbf{máximo} sujeto a
\begin{equation*}
\sum_{i=0}^{n} x_{i} * m_{i} \leq M
\end{equation*}
\begin{equation*}
\land
\end{equation*}
\begin{equation*}
\sum_{i=0}^{n} x_{i} * h_{i} \leq H
\end{equation*}
\end{itemize}
\end{itemize}


\end{enumerate}

\newpage

\begin{flushleft}
\textbf{8. Producción de Gasolina}
\end{flushleft}


\begin{enumerate}[label=\alph*]
\item Proponga un modelo COP para el problema utilizando los valores presentados en la tabla:
\begin{justify}
\begin{itemize}
\item Entradas:
\\
\textbf{Butano = 1, Catalítico Reformado = 2, Nafta Pesado = 3}
\begin{itemize}
\item Lista de tamaño 3 que denota la cantidad de Octano de cada ingrediente:
\begin{equation*}
O_{i} = <120,100,74>
\end{equation*}
\item Lista de tamaño 3 que denota la cantidad de Valor de presión de cada ingrediente:
\begin{equation*}
P_{i} = <60,2.5,4.0>
\end{equation*}
\item Lista de tamaño 3 que denota la cantidad de Volatilidad de cada ingrediente:
\begin{equation*}
V_{i} = <105,3,12>
\end{equation*}
\item Lista de tamaño 3 que denota los costos por galón de cada ingrediente:
\begin{equation*}
C_{i} = <0.58,1.55,0.85>
\end{equation*}
\end{itemize}
\item Salida:
\begin{itemize}
\item Lista de tamaño 3 que denota la cantidad de galones a usar de cada ingrediente:
\begin{equation*}
X_{i} = <x_{0}, x_{1}, x_{2}>
\end{equation*}
Tal que:
\begin{equation*}
\sum_{i = 0}^{2} x_{i} = 12000
\end{equation*}
\newpage
Sujeto a:
\begin{equation*}
(\sum_{i = 0}^{2} x_{i} * o_{i}) \div 12000 \geq 89
\end{equation*}
\begin{equation*}
\land
\end{equation*}
\begin{equation*}
(\sum_{i = 0}^{2} x_{i} * p_{i}) \div 12000 \leq 11
\end{equation*}
\begin{equation*}
\land
\end{equation*}
\begin{equation*}
(\sum_{i = 0}^{2} x_{i} * v_{i}) \div 12000 \geq 17
\end{equation*}
Donde:
\begin{equation*}
\sum_{i = 0}^{2} x_{i} * c_{i}
\end{equation*}
Sea \textbf{mínimo}.
\end{itemize}
\end{itemize}
\end{justify}
\item Presente la implementación del modelo anterior utilizando código MiniZinc:
\item Proponga un modelo genérico asumiendo que los valores en la tabla pueden cambiar dependiendo del
proveedor de cada uno de los ingredientes:
\item Presente la implementación del modelo anterior utilizando código MiniZinc: 
\end{enumerate}

\end{document}