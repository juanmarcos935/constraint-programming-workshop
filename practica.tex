\documentclass[12pt]{article}
\usepackage[spanish]{babel}
\usepackage{amsmath}
\usepackage{amssymb}
\usepackage{graphicx}
\usepackage{ragged2e}
\usepackage{enumitem}
\usepackage{fancyvrb}
\usepackage[official]{eurosym}
\usepackage[utf8x]{inputenc} 
\graphicspath{ {images/} }
\begin{document}
\begin{figure}[t]
\title{Taller 1}
\includegraphics[{width=3cm, height=4cm}]{logouv.jpg}
\end{figure}
\title{Universidad del Valle \\ 
		Escuela de Ingeniería de Sistemas y Computación \\
		Programación por Restricciones \\
		Desarrollo: Taller Modelamiento e Implementación CSPs}
\author{Juan Marcos Caicedo Mej\'ia (1730504-3743)}
\date{Viernes 6 de Diciembre de 2019}
\maketitle

\newpage

\begin{center}
\large
\textbf{PARTE 1 (CSPs)}
\end{center}

\begin{flushleft}
\textbf{3. Secuencia Mágica}
\end{flushleft}

\begin{itemize}
\item Variables:
\begin{itemize}
\item $n$: La variable $n$ (ingresada por el usuario mediante el IDE o proveída por un archivo de datos .dzn), representa la longitud de la secuencia mágica.
\end{itemize}
\item Dominios:
\begin{itemize}
\item $dominio = 0..n-1$: El dominio $dominio$ caracteriza dos cosas: el tamaño de la secuencia mágica (el tamaño del arreglo que la representa) y los posibles valores que toman los números dentro de la secuencia mágica.
\end{itemize}
\item Restricciones:
\begin{itemize}
\item Llamemos $ocurre$ a un predicado que recibe una lista $l$ y un número $x$, así, $ocurre(l,x)$ arroja el número de ocurrencias del número $x$ en la lista $l$. Otro predicado puede ser $posicion$ que dada una lista $l$ y un número $x$, retorna la posición del número $x$ en la lista $l$ (indexando desde 0). La restricción principal del problema consiste en que la secuencia mágica se caracteriza porque el número $i$ ocurre exactamente $x_{i}$ veces en la secuencia. Así, la restricción podría modelarse:
\begin{equation*}
\forall x \in l, ocurre(l, posicion(l, x)) = x
\end{equation*}
Esto quiere decir que el número de la posición de $x$ en la lista $l$ debe ocurrir $x$ veces en la lista $l$.
\item Una restricción redundante adicional, dicta que la suma de todos los números en la secuencia debe sumar el número $n$ (longitud de la secuencia):
\begin{equation*}
\sum_{i=0}^{n-1} x_{i} = n
\end{equation*}
Donde $x_{i}$ es el i-ésimo elemento de la secuencia mágica.
\end{itemize}
\end{itemize}



\end{document}